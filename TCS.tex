\documentclass[]{article}
\usepackage{hyperref}
\usepackage{enumitem}
\usepackage{amsfonts}


\begin{document}
\title{A Tiny Compilation of Spectroscopic Methods}
\author{César Godinho}
\maketitle

\newpage

\begin{abstract}
Nowadays spectroscopy is a very broad topic. From the tiny visible light spectrometer built by Gustav Kirchhoff and Robert Bunsen, in the mid 19th century, to the present day photo-electron spectroscopy with image rendering, that allows us to display images of electron detection mapped spatially.
\par It's not wrong to say that we came a long way from those days, but I felt the urge to write something about a few of the modern (some not so much) techniques. This article is intended to have a "not so formal" aspect, since I'm learning all of the things I write in this document, so I wouldn't consider myself an expert. That said, here is a disclaimer of everything you might read on this document: don't trust anything that is possibly too suspicious, also I will try to cite most of the sources I use. \par However, it's not as easy as it sounds to get all the books or excerpts I read in here, so don't get mad at me. Just to reiterate, I'm doing this primarily for fun, so if you intend to learn these further discussed topics properly please \textbf{don't use this miserable document} (that also has no revision whatsoever). Enough said, lets stop with the excuses and get into it.
\end{abstract}

\newpage

\section{Introduction}
In this first section are described the techniques that will be discussed later. If the reader was expecting a section about the history of spectroscopy (than you probably are not like me, I skip all those sections all the time) there are plenty of sources online about the subject, like the trustworthy Wikipedia page about it \href{https://en.wikipedia.org/wiki/History_of_spectroscopy}{here}.

\subsection{Subjects discussed}
\par There are a lot of complementary and modified techniques that can be studied nowadays in the world of spectroscopy, the ones that will be referenced in here, that are not necessarily the most "important" nor by any significant order in the following list, are the ones I consider the most interesting to "take a look at". Some of them I don't know much about, but I will try to be as complete and correct as possible.
\par After enumerating all the topics (instead of having a list of contents, I know, don't judge me, maybe at the time of the reading there is already one here), it's going to be described what will be the line of thought inside of each topic and how to approach the structure of each one.\\

The covered spectroscopic techniques will be:

\begin{itemize}
\item[$\to$] X-Ray Fluorescence Spectroscopy (XRF)
\item[$\to$] X-Ray Photo-electron Spectroscopy (XPS)
\item[$\to$] Auger Electron Spectroscopy (AES)
\item[$\to$] Raman Spectroscopy
\item[$\to$] Infrared Spectroscopy
\item[$\to$] Electron Spin Resonance Spectroscopy (ESR)
\item[$\to$] Ultraviolet and Visible Light Spectroscopy (UV-Vis)
\end{itemize}

\par To understand some of the physical phenomenons behind these techniques, it will be required to discuss briefly about some "theoretical" (simplified...) concepts, for example for Raman and Infrared Spectroscopy it is useful to know about rotational-vibrational spectra, etc. It doesn't seem like much, but these techniques are far from few to analyze in here.

\subsection{Concepts to know}
\par In order to understand all of the discussed concepts the reader is advised to have some extended concepts of math, some basic concepts of physics, majorly molecular and atomic physics. Some of these subjects will be slightly explained and, most likely, cited from other sources. However, there are no such thing as theory only sections that explain deeply all the used concepts, which are explained "on the fly" for most of the this file. Excepting for a few atomic and molecular considerations in the next chapter. 

\subsection{Further reading}
Although there were many cited documents and papers, it is notable to say here, that one of the must read book, that was undoubtly used to compile this document, is the \textit{Modern Spectroscopy by J. M. Hollas} \cite{JMH_Modern}. A great book for an extensive reading in the field, and something I couldn't even aspire to write nor match.

\subsection{Brief description}
A brief description of each method listed above is given here, so the reader can see what interests him/her the most and skip right away to the hot stuff.

\subsubsection{X-Ray Fluorescence (XRF)}
This technique covers the readout of X-Ray emission coming from elements, atomic or not, that are bombarded with high energy radiation (X-Rays, generally higher then the emitted photon), and by interacting with it in a specific way, produce a very characteristic radiation energy, that can be used to analyze a lot of materials (metals, ceramics, ...) qualitatively and quantitatively. 
\par It can also even be used to some more exotic tasks, such as film thickness measurement. Because X-Rays are not attenuated very easily in the air, vacuum is not strictly necessary to perform measurements with this technique.\\

Some of its pros and cons \cite{GI_XRF}:
\begin{itemize}
\item[\checkmark] Cheap Equipment
\item[\checkmark] Possibility of creating mobile devices for \textit{in situ} measurements
\item[\checkmark] Large range of studiable materials
\item[$\times$] Poor spot resolution (there are workarounds)
\item[$\times$] Larger limit of detection than other methods
\item[$\times$] More destructive than other methods
\end{itemize}

\subsubsection{X-Ray Photo-electron Spectroscopy (XPS)} %TODO

\newpage

\begin{thebibliography}{9}
\bibitem{JMH_Modern} 
J. Michael Hollas. 
\textit{Modern Spectroscopy}. 
John Wiley and Sons Ltd., 4th Ed., 2004.

\bibitem{GI_XRF} 
X-ray fluorescence spectroscopy,
\\\texttt{https://www.gems-inclusions.com/inclusions-studies/analytical-methods/xrfs-2/}
\end{thebibliography}

\end{document}